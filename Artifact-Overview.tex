% Options for packages loaded elsewhere
\PassOptionsToPackage{unicode}{hyperref}
\PassOptionsToPackage{hyphens}{url}
%
\documentclass[
]{article}
\author{}
\date{}

\renewcommand{\topfraction}{1} %default: 0.7
\renewcommand{\bottomfraction}{1} %default: 0.3
\renewcommand{\textfraction}{0.1} %default: 0.2
\renewcommand{\floatpagefraction}{1} %default: 0.6

\usepackage{fontspec}
\setmainfont{Liberation Serif}
\setmonofont{Inconsolata}
\usepackage{amsmath,amssymb}
\usepackage{verbatimbox}
\usepackage{fullpage}
\usepackage{iftex}
\ifPDFTeX
  \usepackage[T1]{fontenc}
  \usepackage[utf8]{inputenc}
  \usepackage{textcomp} % provide euro and other symbols
\else % if luatex or xetex
  \usepackage{unicode-math}
  \defaultfontfeatures{Scale=MatchLowercase}
  \defaultfontfeatures[\rmfamily]{Ligatures=TeX,Scale=1}
\fi
% Use upquote if available, for straight quotes in verbatim environments
\IfFileExists{upquote.sty}{\usepackage{upquote}}{}
\IfFileExists{microtype.sty}{% use microtype if available
  \usepackage[]{microtype}
  \UseMicrotypeSet[protrusion]{basicmath} % disable protrusion for tt fonts
}{}
\makeatletter
\@ifundefined{KOMAClassName}{% if non-KOMA class
  \IfFileExists{parskip.sty}{%
    \usepackage{parskip}
  }{% else
    \setlength{\parindent}{0pt}
    \setlength{\parskip}{6pt plus 2pt minus 1pt}}
}{% if KOMA class
  \KOMAoptions{parskip=half}}
\makeatother
\usepackage{xcolor}
\IfFileExists{xurl.sty}{\usepackage{xurl}}{} % add URL line breaks if available
\IfFileExists{bookmark.sty}{\usepackage{bookmark}}{\usepackage{hyperref}}
\hypersetup{
  hidelinks,
  pdfcreator={LaTeX via pandoc}}
\urlstyle{same} % disable monospaced font for URLs
\setlength{\emergencystretch}{3em} % prevent overfull lines
\providecommand{\tightlist}{%
  \setlength{\itemsep}{0pt}\setlength{\parskip}{0pt}}
\setcounter{secnumdepth}{-\maxdimen} % remove section numbering
\ifLuaTeX
  \usepackage{selnolig}  % disable illegal ligatures
\fi

\begin{document}

\hypertarget{getting-started}{%
\section{Getting Started}\label{getting-started}}

We show here the simplest way to (dry) run the artifact, which is via
\href{https://docs.docker.com/get-docker/}{Docker}. The alternatives are
explained in the
\protect\hyperlink{comments-on-setting-up-environment}{comments to
setting up the environment}. Every step shouldn't take longer than 5
minutes.

\begin{enumerate}
\def\labelenumi{\arabic{enumi}.}
\item
  Extract the artifact tarball somewhere on the disk, and open a shell
  session in the root directory. The \texttt{ls} (accordingly,
  \texttt{dir} on Windows) should show you:

\begin{verbnobox}[\small]
Stability    pkgs    start    Overview.pdf    shell.nix    top-1000-pkgs.txt
\end{verbnobox}
\item
  From here, run Docker as follows:

\begin{verbnobox}[\small]
docker run --rm -it -v "$PWD":/artifact nixos/nix sh -c "cd artifact; nix-shell --pure"
\end{verbnobox}

  Note: running Docker may require root privileges. Other notes from the
  \protect\hyperlink{steps-to-setup}{Steps To Setup} section also apply.
\item
  Make sure (e.g.~with \texttt{ls}) that you are in the artifact root
  (e.g.~\texttt{ls} shows the \texttt{Stability} directory among
  others). Pull in dependencies of our code:

\begin{verbnobox}[\small]
JULIA_PROJECT=Stability julia -e 'using Pkg; Pkg.instantiate()'
\end{verbnobox}
\item
  To run our type stability analysis on one simple Julia package
  \texttt{Multisets}, do the following:

  \begin{enumerate}
  \def\labelenumii{\alph{enumii}.}
  \item
    Enter \texttt{start} directory and process the package:

\begin{verbnobox}[\small]
cd start
../Stability/scripts/proc_package.sh "Multisets 0.4.12"
\end{verbnobox}

    There should be a number of new files with raw data in
    \texttt{start/Multisets},
    e.g.

\begin{verbnobox}[\small]
stability-stats-per-instance.csv
\end{verbnobox}

  \item
    Convert the raw data into a CSV file:

\begin{verbnobox}[\small]
../Stability/scripts/pkgs-report.sh
\end{verbnobox}

    There should be a new file \texttt{start/report.csv}.
  \item
    Generate tables:

\begin{verbnobox}[\small]
JULIA_PROJECT=../Stability ../Stability/scripts/tables.jl
\end{verbnobox}

    You should see stability data about the Multisets package as an
    output:

\begin{verbnobox}[\small]
 Table 1
 3×3 DataFrame
  Row │ Stats      Stable   Grounded 
      │ String     Float64  Float64  
 ─────┼──────────────────────────────
    1 │ Mean          73.0      55.0
    2 │ Median        73.0      55.0
    3 │ Std. Dev.      0.0       0.0
\end{verbnobox}

\begin{verbnobox}[\small]
 Table 2
 1×7 DataFrame
  Row │ package    Methods  Instances  Inst/Meth  Varargs (%)  Stable (%)  Grounded (%) 
      │ String     Int64    Int64      Float64    Int64        Int64       Int64        
 ─────┼─────────────────────────────────────────────────────────────────────────────────
    1 │ Multisets        7         11        1.6            0          73            55
\end{verbnobox}

    Note that the format is the same as Tables 1 and 2 of the paper,
    only the numbers are computed over one small package.
  \item
    Generate plots:

\begin{verbnobox}[\small]
JULIA_PROJECT=../Stability julia -L ../Stability/scripts/plot.jl -e 'plot_pkg("Multisets")'
\end{verbnobox}

    There should be a group of figures created under
    \texttt{Multisets/figs}. The ones that correspond to Figure 10 of
    the paper are

\begin{verbnobox}[\small]
Multisets/figs/Multisets-size-vs-stable.pdf
Multisets/figs/Multisets-size-vs-grounded.pdf
\end{verbnobox}

    You should be able to browse the figures using your host system's
    PDF viewer (all files created inside Docker container under the
    \texttt{/artifact} directory will be visible in your host file
    system).\\
    The figures should be similar to ones in
    \texttt{start/Multisets/figs-ref} (provided for reference).
  \end{enumerate}
\item
  To shut down the Docker session, run \texttt{exit}.
\end{enumerate}

\hypertarget{list-of-claims-from-paper-supported-by-artifact}{%
\section{List of Claims From Paper Supported by
Artifact}\label{list-of-claims-from-paper-supported-by-artifact}}

This artifact aims to give exact directions on how to reproduce experiments reported in
Section 5 (Empirical Study) of the Type Stability in Julia paper
submitted to OOPSLA '21; in particular:
  \begin{itemize}
  \item
    Table 1,
  \item
    Table 2,
  \item
    Figure 10.
  \end{itemize}

\section{Contents of Artifact}

Figure 1 describes the contents of our artifact. Some of the scripts and source files
have instructional comments inside. We submit the Git history of the project:
the submitted version lives on the \texttt{artifact} branch. The same repository
is available on Github
({\color{blue}\href{https://github.com/ulysses4ever/julia-type-stability}{ulysses4ever/julia-type-stability}})
but without the \texttt{pkgs} directory: this directory contains data
about the exact Julia package versions we used to compute statistics, and also
the full raw data produced by our analysis, which you can use to build the final
tables without having to run the experiment on all 1000 packages. This directory
takes 1.6 Gb space when uncompressed, the rest of the artifact fits into several
megabytes including the Git history.

\begin{figure}[ht]
\label{fig:contents}
\begin{verbnobox}[\small]
├── shell.nix - system dependencies via Nix
├── Stability - our Julia package to analyze type stability
│  ├── Manifest.toml - machine description of our Julia package (1)
│  ├── Project.toml  - machine description of our Juila package (2)
│  ├── scripts
│  │  ├── proc_package.sh - run stability analysis on one package (produce raw analysis data)
│  │  ├── proc_package_parallel.sh - run stability analysis on several packages (batch mode)
│  │  ├── plot.jl - using raw data generate figures like Figure 10
│  │  ├── pkgs-report.sh - using raw data generate aggregate report (report.csv)
│  │  └── tables.jl - using report.csv render Tables 1 and 2 of the paper
│  ├── src
│  │  ├── pkg-test-override.jl - override Julia's test suite functionality to hook in our analysis
│  │  └── Stability.jl - code analyzing type stability
│  ├── startup.jl - convenience script to start Julia with our package loaded (1)
│  └── startup.sh - convenience script to start Julia with our package loaded (2)
├── pkgs
│  ├── ... - 1000 packages with descriptions ({Project, Manifest}.toml) and raw data precomputed
│  └── report.csv - report on 1000 packages used to generate tables and figures in the paper
├── start
│  └── Multisets - example package for the Getting Started phase
│     ├── figs-ref - reference figures for the Getting Started phase
│     ├── Manifest.toml - machine description of the package (1)
│     └── Project.toml - machine description of the package (2)
├── top-1000-pkgs.txt - list of 1000 most-starred Julia project
├── Overview.pdf - this file
├── .git - Git history of this project
└── .gitignore - we ignore the pkgs directory (see above)
\end{verbnobox}
\caption{Contents of Artifact}
\end{figure}

\hypertarget{step-by-step-instructions}{%
\section{Step By Step Instructions}\label{step-by-step-instructions}}

There are three parts to the rest of this document:

\begin{enumerate}
\def\labelenumi{\arabic{enumi}.}
\item
  Metadata about our artifact and experiment as suggested by AEC
  Guidelines.
\item
  Discussion of environment setup.
\item
  Reproducing results from Section 5 of the paper.
\end{enumerate}

\hypertarget{metadata}{%
\subsection{Metadata}\label{metadata}}

\hypertarget{hardware-and-timings}{%
\paragraph{Hardware and Timings}\label{hardware-and-timings}}

AEC guidelines suggest adding approximate timings for all operations.

First, our hardware specs. We run experiments in a virtualized
environment on server hardware:

\begin{itemize}
\item
  32 Intel (Skylake) CPUs --- speed up our experiments quite a bit
  (nearly linear thanks to GNU parallel)
\item
  64 Gb RAM --- shouldn't be critical to the experiment, but the more
  CPUs you have the more memory you need (our 32 CPUs consumed 15 Gb at
  some point)
\item
  1.5 Tb disk space --- same as RAM, shouldn't be critical but be sure to
  provision several Gb per processor.
\end{itemize}

Using that hardware we get:

\begin{itemize}
\tightlist
\item
  Reproducing Table 1 (1000 packages): 7 hrs
\item
  Reproducing Table 2 (10 packages): 20 mins
\item
  Reproducing Fig. 10 (heat maps): \textless{} 5 minutes
\end{itemize}

These numbers, again, are heavily dependent on the number of CPUs
available on the system.

\hypertarget{expected-warnings}{%
\paragraph{Expected Warnings}\label{expected-warnings}}

AEC guidelines suggest mentioning expected warnings in the README.

\begin{enumerate}
\def\labelenumi{\arabic{enumi}.}
\item
  During package analysis (e.g.~reproducing Tables 1 and 2) Julia warns
  about us overriding its test-suite functionality. This is fine because
  that appeared to be the most convenient way of gathering statistics we
  need.

\begin{verbnobox}[\small]
WARNING: Method definition gen_test_code##kw(Any, typeof(Pkg.Operations.gen_test_code), String)
  in module Operations at /buildworker/.../julia/stdlib/v1.5/Pkg/src/Operations.jl:1316
  overwritten in module Stability at /artifact/Stability/src/pkg-test-override.jl:6.
  ** incremental compilation may be fatally broken for this module **
\end{verbnobox}
\item
  GNU parallel wants to be cited and prints the following message unless
  you run it dry with the \texttt{-\/-citation} flag first. It is safe
  to ignore.

\begin{verbnobox}[\small]
Academic tradition requires you to cite works you base your article on.
If you use programs that use GNU Parallel to process data for an article
...
Come on: You have run parallel 13 times. Isn't it about time 
you run 'parallel --citation' once to silence the citation notice?
\end{verbnobox}
\item
  The main body of the experiment for Tables 1 and 2 uses Julia Package
  manager to get and run tests of various Julia packages. The package
  manager as well as test suites are rather verbose, so expect a lot of
  text on the screen. Some tests will fail, but this is expected: the
  idea is that no matter what the outcome of the test is, some Julia code
  will be compiled during the test, and that is the code that we analyze.

  Some test suites among the 1000 packages in our data set will fail fatally for
  various reasons (usually, dependencies-related). As noted in the
  paper, Section 5.1 (Methodology), out of the 1000, over seven hundred test
  suites do produce results that we can analyze.
\end{enumerate}

\hypertarget{comments-on-setting-up-environment}{%
\subsection{Comments on Setting Up
Environment}\label{comments-on-setting-up-environment}}

Our artifact has modest dependencies, and it should be possible to
manually install them on a different machine; however, it is not necessary
to do so because we provide two other essential
pieces of infrastructure: a) automatic dependency management (via the
Nix package manager), b) isolation from the host system (via Docker).
Still, both of these pieces are optional.

\textit{Note: if the Docker approach from Getting Started section works good for
you, it's not necessary to follow this section.}

We first expand a little more on the setup and then provide the sequence
of steps that get you in the right environment.

\hypertarget{dependencies}{%
\paragraph{Dependencies}\label{dependencies}}

\begin{itemize}
\item
  Bash shell
\item
  Julia v1.5.4
\item
  GNU parallel
\item
  \texttt{timeout} from GNU coreutils
\end{itemize}

\hypertarget{dependency-management-via-nix-and-reproducibility}{%
\paragraph{Dependency Management via Nix and
Reproducibility}\label{dependency-management-via-nix-and-reproducibility}}

To make sure the versions of dependencies are the same as we used, and
also to automate the process of getting them, we use the
\href{https://nixos.org/manual/nix/stable/\#chap-introduction}{Nix
package manager} -- a general package manager
\href{https://nixos.org/guides/install-nix.html}{available} on Linux,
MacOS and Windows (via WSL).

The \texttt{shell.nix} file in the root of the artifact describes the
dependencies we have. If the user has Nix installed and doesn't need
isolation (e.g.~believes that we don't \texttt{rm\ -rf} their home
directory), they can use it directly without Docker and skip the
section on Docker.

There is another kind of dependencies, though: the analyzed Julia
packages, which are not covered by Nix or any other (known to us)
general package manager. To get those, we use the Julia package manager
\texttt{Pkg} (part of the Julia dependency). To generate figures from
the paper as close as possible during the experiment, we supply the
metadata about versions of the packages we analyzed. These metadata are
stored in \texttt{pkgs} and \texttt{start} directories of the artifact.
They have a set of subdirectories named after particular packages we
analyzed, and every package subdirectory contains two files related to
\texttt{Pkg}: \texttt{Project.toml} and \texttt{Manifest.toml}.
Supplying these ensures that you will run the analysis on the same versions of
the packages as we did, but it does not guarantee 100\% reproducibility of
the reported numbers because test suites of some analyzed packages contain
non-determinism (naturally).

\hypertarget{isolation-plus-nix-dependency-via-docker}{%
\paragraph{Isolation Plus Nix Dependency via
Docker}\label{isolation-plus-nix-dependency-via-docker}}

We suggest using Docker to avoid unexpected interactions with the host
system. This also gets you Nix. In theory, you could use a Docker image
without Nix and get all the dependencies there manually, or we could supply
a custom Docker image with them (this would cost several hundreds
megabytes in the artifact's real estate). But we don't discuss these
configurations because Docker+Nix seems like a strictly better option.

Note that insider Docker, the commands are run under the root user. This
makes life easier in one aspect: Julia package manager stores a lot of
metadata and makes some of it inaccessible for cleanup in the end. This
is no problem under reboot but if running outside Docker and without root
privileges, this will eat up some disk space, proportional to the number
of packages analyzed. For 1000 packages this can add up to hundreds of
gigabytes. If you don't have that much disc space, run either in Docker
or under root.

\hypertarget{steps-to-setup}{%
\paragraph{Steps To Setup}\label{steps-to-setup}}

We mark the first two steps as optional, but the simplest and most-likely-to-succeed
path is to apply both of them.
Alternatively, you can manually install the dependencies listed at the beginning
of this section, open artifact root in the shell, and go to step 3.

\begin{enumerate}
\def\labelenumi{\arabic{enumi}.}
\item
  {[}Optional, apply if Isolation is desired{]}\\
  Start a Docker container with Nix installed by executing the following
  in the root directory of the artifact:

  \texttt{docker\ run\ -\/-rm\ -it\ -v\ "\$PWD":/artifact\ nixos/nix\ sh\ -c\ "cd\ artifact"}

  Note 1: running the \texttt{docker} command may require root
  privileges on your system. We assume that you know how to use Docker
  on your system.

  Note 2: \texttt{"\$PWD"} expands to the path of the current directory
  in most shells. You can check it with \texttt{echo\ "\$PWD"}. If not,
  please use equivalent command for your shell or type in the path
  manually.

  Note 3 for Windows and MacOS users: note that Docker is sometimes
  reported to be slow with file operations concerning bind mounts
  (that's the \texttt{-v} flag) on these platforms. If you happen to use
  them and want to reproduce the experiment on 1000 packages, you may be
  better off with applying this manually inside a virtual machine with
  some Linux installation. Getting a Linux VM is not covered here
  because it's assumed to be a common knowledge.
\item
  {[}Optional, apply if Dependency Management is desired{]}\\
  Make sure the artifact files are in the current directory in your
  shell session (e.g.~by executing \texttt{ls} in the shell). From
  there, enter the Nix shell by executing \texttt{nix-shell\ -\/-pure}.
  Nix shell is a Bash shell that has all the dependencies specified in
  \texttt{shell.nix} visible (i.e.~in \texttt{\$PATH}).
\item
  {[}Sanity Checking{]} Make sure you have Julia and GNU parallel up:
  \texttt{julia\ -\/-version\ \&\&\ parallel\ -\/-version}.
  Current directory should be artifact's root (e.g. \texttt{ls} should show
  Stability directory, shell.nix file, etc.).
\item
  Pull in dependencies of our Julia code:

\begin{verbnobox}[\small]
JULIA_PROJECT=Stability julia -e 'using Pkg; Pkg.instantiate()'
\end{verbnobox}
\end{enumerate}

\hypertarget{reproducing-results-from-section-5}{%
\subsection{Reproducing Results from Section
5}\label{reproducing-results-from-section-5}}

As noted in the Timings section, reproducing Table 1 requires a lot of
computational resources to analyze 1000 Julia packages. We suggest starting
from only top 10 packages, which also provide data for Table 2. In the end of
this section, we show how to process all 1000 packages (or less).

Instructions below assume the environment is set up, e.g. by applying the first 3
steps from the Getting Started section or following the discussion in the
Comments on Setup section above.

\begin{enumerate}
  \def\labelenumii{\alph{enumii}.}
  \item
    Enter the \texttt{pkgs} directory in the root of the artifact and run the analysis
    on 10 packages:

\begin{verbnobox}[\small]
cd pkgs
../Stability/scripts/proc_package_parallel.sh ../top-1000-pkgs.txt 10
\end{verbnobox}

    This takes 20 minutes in our fully parallel setup (\# CPUs \textgreater{}
    10). This script (1) runs the analysis producing the raw data per package, and
    (2) runs the reporting script; this is equivalent is steps 4.a and
    4.b from the Getting Started section. The resulting \texttt{report.csv}
    in the current directory should get a new time stamp.

  \item
    Generate both tables as before:

\begin{verbnobox}[\small]
JULIA_PROJECT=../Stability ../Stability/scripts/tables.jl
\end{verbnobox}

    Table 1 will have significantly different numbers than Table 1 in the paper
    because we analyzed only 10 packages, not 1000.

    Table 2 should have mostly the same numbers as in the paper. Slight
    differencies in several lines are due to inherent non-determinism in the
    test suites of respective packages.

  \item
    Reproducing Figure 10 of the paper:

\begin{verbnobox}[\small]
JULIA_PROJECT=../Stability julia -L ../Stability/scripts/plot.jl -e 'plot_pkg("Pluto")'
\end{verbnobox}

    There should be a group of figures created under
    \texttt{Pluto/figs}. The ones that we have on Figure 10 are

\begin{verbnobox}[\small]
Pluto/figs/Pluto-size-vs-stable.pdf
Pluto/figs/Pluto-size-vs-grounded.pdf
\end{verbnobox}

\end{enumerate}

Step 1 can be adjusted to run either on the whole 1000-packages set by omitting
the $10$ argument to \texttt{proc\_package\_parallel.sh} or to run on first
$n$ packages by replacing $10$ with $n$.

\end{document}
